% \iffalse
\let\negmedspace\undefined
\let\negthickspace\undefined
\documentclass[journal,12pt,twocolumn]{IEEEtran}
\usepackage{cite}
\usepackage{amsmath,amssymb,amsfonts,amsthm}
\usepackage{algorithmic}
\usepackage{graphicx}
\usepackage{textcomp}
\usepackage{xcolor}
\usepackage{txfonts}
\usepackage{listings}
\usepackage{enumitem}
\usepackage{mathtools}
\usepackage{gensymb}
\usepackage{comment}
\usepackage{tikz}
\usepackage[breaklinks=true]{hyperref}
\usepackage{tkz-euclide} 
\usepackage{listings}
\usepackage{gvv}
\def\inputGnumericTable{}
\usepackage[latin1]{inputenc}                              
\usepackage{color}                                            
\usepackage{array}                                            
\usepackage{longtable}                                       
\usepackage{calc}                                             
\usepackage{multirow}                                         
\usepackage{hhline}                                           
\usepackage{ifthen}                                           
\usepackage{lscape}

\newtheorem{theorem}{Theorem}[section]
\newtheorem{problem}{Problem}
\newtheorem{proposition}{Proposition}[section]
\newtheorem{lemma}{Lemma}[section]
\newtheorem{corollary}[theorem]{Corollary}
\newtheorem{example}{Example}[section]
\newtheorem{definition}[problem]{Definition}
\newcommand{\BEQA}{\begin{eqnarray}}
\newcommand{\EEQA}{\end{eqnarray}}
\newcommand{\define}{\stackrel{\triangle}{=}}
\theoremstyle{remark}
\newtheorem{rem}{Remark}
\begin{document}

\bibliographystyle{IEEEtran}
\vspace{3cm}

\title{DISCRETE 11.9.4 Q-1}
\author{EE23BTECH11207 -KAILASH.C$^{*}$% <-this % stops a space
}
\maketitle
\newpage
\bigskip

\renewcommand{\thefigure}{\theenumi}
\renewcommand{\thetable}{\theenumi}


\textbf{QUESTION:}\\
Find the sum of n terms of the series:
$1\times2+2\times3+3\times4+4\times5+....$\\ \\
\textbf{SOLUTION:}\\
In the above series,we have:$a_n=n(n+1)$ where n is $n_{th}$ and $n+1$ is ${n+1}_{th}$element of the natural number correspondingly and it starts from zero.
\begin{align}
     S&=\sum_{n=0}^{n} n \cdot (n+1)\\
     &=\sum_{n=0}^{n} (n^2 + n)\\
     &=\sum_{n=0}^{n} n^2 + \sum_{n=0}^{n}
\end{align}
By the formula for sum of series,we have:
\begin{align}
    \sum_{n=0}^{n} n^2&=\frac{n(n+1)(2n+1)}{6}\ \\
    \sum_{n=0}^{n} n&=\frac{n(n+1)}{2}\
\end{align}
Using eq-(4) and eq-(5) in eq-(3):
\begin{align}
    S&=\frac{n(n+1)(2n+1)}{6} + \frac{n(n+1)}{2}\ \\
    &=\frac{n(n+1)}{6} (2n+1 + 3)\ \\
    &=\frac{n(n+1)(2n+4)}{6}\\
    &=\frac{n(n+1)(n+2)}{3}
\end{align}\\
\textbf{Z-Transformation of S(n):}\\
\begin{align}
S(z)&=\sum_{n=0}^{\infty} \frac{n(n+1)(n+2)}{3} \cdot z^{-n}\\
&=\frac{1}{3} \sum_{n=0}^{\infty} n(n+1)(n+2) \cdot z^{-n}\\
&=\frac{1}{6} \sum_{n=0}^{\infty} (n^3 + 3n^2 + 2n) \cdot z^{-n}
\end{align}
By applying z-transformation to each term:
\begin{align}
    S(z)&=\frac{1}{3}\left(\sum_{n=0}^{\infty}n^3 z^{-n}+\sum_{n=0}^{\infty}3n^2z^{-n}+\sum_{n=0}^{\infty}2nz^{-n}\right)
\end{align}
Let:
\begin{align}
    X_1(z)&=\sum_{n=0}^{\infty}n^3 z^{-n}\\ X_2(z)&=\sum_{n=0}^{\infty}3n^2z^{-n}\\ X_3(z)&=\sum_{n=0}^{\infty}2nz^{-n}
\end{align}
By the z-transformation formulas,we have:
\begin{align}
     Z[n^k]&=-Z\frac{d}{dZ}Z[n^{k-1}]
\end{align}
By using eq-(17) in eq-(14),(15) and (16) and solving,we get:
\begin{align}
    X_1(z) &= z^2 \left( \frac{1}{2} + \frac{1}{2}(z - 1)^{-1} + \frac{1}{4}(z - 1)^{-2} \right)\\
    X_2(z) &= -3 \left( \frac{1}{(z - 1)^2} + \frac{1}{2}(z - 1)^{-3} \right) \\
    X_3(z) &= -2 \left( \frac{1}{(z - 1)^2} + (z - 1)^{-3} \right)
\end{align}
By combining eq-(18),(19) and (20) we get S(z):
\begin{align}
S(z) &= \frac{1}{3} \left( \frac{z^2}{2} + \frac{z^2}{2}(z - 1)^{-1} + \frac{z^2}{4}(z - 1)^{-2}  - 9 \right. \nonumber \\
&\quad\left( \frac{1}{(z - 1)^2} + \frac{1}{2}(z - 1)^{-3} \right) + 2z \left. \left( -2 \left( \frac{1}{(z - 1)^2} + (z - 1)^{-3} \right) \right) \right)
\end{align}
The Z-transformation of eq-(8) is eq-(21) with r.o.c as:$|z|>1$
\newpage
\textbf{Graph of S(n) vs n:}
\begin{figure}[h]
        \centering
\includegraphics[width=1.0\linewidth]{S(n)_plot.png}
    \end{figure}
\end{document}
