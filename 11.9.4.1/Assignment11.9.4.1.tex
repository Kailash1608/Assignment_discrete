% \iffalse
\let\negmedspace\undefined
\let\negthickspace\undefined
\documentclass[journal,12pt,twocolumn]{IEEEtran}
\usepackage{cite}
\usepackage{amsmath,amssymb,amsfonts,amsthm}
\usepackage{algorithmic}
\usepackage{graphicx}
\usepackage{textcomp}
\usepackage{xcolor}
\usepackage{txfonts}
\usepackage{listings}
\usepackage{enumitem}
\usepackage{mathtools}
\usepackage{gensymb}
\usepackage{comment}
\usepackage{tikz}
\usepackage[breaklinks=true]{hyperref}
\usepackage{tkz-euclide} 
\usepackage{listings}
\usepackage{gvv}
\def\inputGnumericTable{}
\usepackage[latin1]{inputenc}                              
\usepackage{color}                                            
\usepackage{array}                                            
\usepackage{longtable}                                       
\usepackage{calc}                                             
\usepackage{multirow}                                         
\usepackage{hhline}                                           
\usepackage{ifthen}                                           
\usepackage{lscape}

\newtheorem{theorem}{Theorem}[section]
\newtheorem{problem}{Problem}
\newtheorem{proposition}{Proposition}[section]
\newtheorem{lemma}{Lemma}[section]
\newtheorem{corollary}[theorem]{Corollary}
\newtheorem{example}{Example}[section]
\newtheorem{definition}[problem]{Definition}
\newcommand{\BEQA}{\begin{eqnarray}}
\newcommand{\EEQA}{\end{eqnarray}}
\newcommand{\define}{\stackrel{\triangle}{=}}
\theoremstyle{remark}
\newtheorem{rem}{Remark}
\begin{document}

\bibliographystyle{IEEEtran}
\vspace{3cm}

\title{DISCRETE 11.9.4 Q-1}
\author{EE23BTECH11207 -KAILASH.C$^{*}$% <-this % stops a space
}
\maketitle
\newpage
\bigskip

\renewcommand{\thefigure}{\theenumi}
\renewcommand{\thetable}{\theenumi}


\textbf{QUESTION:}\\
Find the sum of n terms of the series:
$1\times2+2\times3+3\times4+4\times5+....$\\ \\
\textbf{SOLUTION:}\\
From observing the series above,We can say that is a sum of the series:$\sum_{n=0}^{\infty} n \cdot (n+1)$
\begin{align}
     S&=\sum_{n=0}^{\infty} n \cdot (n+1)\\
     S&=\sum_{n=0}^{\infty} (n^2 + n)\\
     S&=\sum_{n=0}^{\infty} n^2 + \sum_{n=0}^{\infty} n\ \\
\end{align}
By the formula for sum of series,we have:
\begin{align}
    \sum_{n=0}^{\infty} n^2&=\frac{n(n+1)(2n+1)}{6}\ \\
    \sum_{n=0}^{\infty} n&=\frac{n(n+1)}{2}\
\end{align}
Using eq-(5) and eq-(6) in eq-(3):
\begin{align}
    S&=\frac{n(n+1)(2n+1)}{6} + \frac{n(n+1)}{2}\ \\
    S&=\frac{n(n+1)}{6} (2n+1 + 3)\ \\
    S&=\frac{n(n+1)(2n+4)}{6}\
\end{align}\\
\textbf{ANSWER:}\\
The sum of b terms of series is:$S=\frac{n(n+1)(2n+4)}{6}$\\ \\
\textbf{Z-Transformation of S(n):}\\
\begin{align}
S(z)&=\sum_{n=0}^{\infty} \frac{n(n+1)(2n+4)}{6} \cdot z^{-n}\\
S(z)&=\frac{1}{6} \sum_{n=0}^{\infty} n(n+1)(2n+4) \cdot z^{-n}\\
S(z)&=\frac{1}{6} \sum_{n=0}^{\infty} (2n^3 + 6n^2 + 4n) \cdot z^{-n}
\end{align}
By applying z-transformation to each term:
\begin{align}
    S(z)&=\frac{1}{6}\left(\sum_{n=0}^{\infty}2n^3 \cdotz^{-n}+\sum_{n=0}^{\infty}6n^2\cdotz^{-n}+\sum_{n=0}^{\infty}4n\cdotz^{-n}\right)
\end{align}
By the z-transformation formulas,we have:
\begin{align}
     Z\{n^k\}&=\frac{1}{(1-z^{-1})^{k+1}}\\
     Z\{n\}&=\frac{z^{-1}}{(1-z^{-1})^2}
\end{align}
Using eq-(14) and eq-(15) in eq-(13):
\begin{align}
    S(z)&=\frac{1}{6} \left( 2 \sum_{n=0}^{\infty} n^3 \cdot z^{-n} + 6 \sum_{n=0}^{\infty} n^2 \cdot z^{-n} + 4 \sum_{n=0}^{\infty} n \cdot z^{-n} \right)\\
    S(z)&=\frac{1}{6} \left( 2 \cdot \frac{z^{-1}}{(1-z^{-1})^4} + 6 \cdot \frac{z^{-1}}{(1-z^{-1})^3} + 4 \cdot \frac{z^{-1}}{(1-z^{-1})^2} \right)\\
    S(z)&=\frac{1}{6} \left( \frac{2z^{-1}}{(1-z^{-1})^4} + \frac{6z^{-1}}{(1-z^{-1})^3} + \frac{4z^{-1}}{(1-z^{-1})^2} \right)
\end{align}
The Z-transformation of eq-(9) as eq-(18) with r.o.c as:$|z|>1$
\newpage
\textbf{Graph of S(n) vs n:}
\begin{figure}[h]
        \centering
\includegraphics[width=1.0\linewidth]{S(n)_plot.png}
    \end{figure}
\end{document}
