% \iffalse
\let\negmedspace\undefined
\let\negthickspace\undefined
\documentclass[journal,12pt,twocolumn]{IEEEtran}
\usepackage{cite}
\usepackage{amsmath,amssymb,amsfonts,amsthm}
\usepackage{algorithmic}
\usepackage{graphicx}
\usepackage{textcomp}
\usepackage{xcolor}
\usepackage{txfonts}
\usepackage{listings}
\usepackage{enumitem}
\usepackage{mathtools}
\usepackage{gensymb}
\usepackage{comment}
\usepackage{tikz}
\usepackage[breaklinks=true,hidelinks]{hyperref}
\usepackage{tkz-euclide} 
\usepackage{listings}
\usepackage{gvv}
\def\inputGnumericTable{}
\usepackage[latin1]{inputenc}                              
\usepackage{color} 
\usepackage{array}                                            
\usepackage{longtable}                                       
\usepackage{calc}                                             
\usepackage{multirow}                                         
\usepackage{hhline}                                           
\usepackage{ifthen}                                           
\usepackage{lscape}

\newtheorem{theorem}{Theorem}[section]
\newtheorem{problem}{Problem}
\newtheorem{proposition}{Proposition}[section]
\newtheorem{lemma}{Lemma}[section]
\newtheorem{corollary}[theorem]{Corollary}
\newtheorem{example}{Example}[section]
\newtheorem{definition}[problem]{Definition}
\newcommand{\BEQA}{\begin{eqnarray}}
\newcommand{\EEQA}{\end{eqnarray}}
\newcommand{\define}{\stackrel{\triangle}{=}}
\theoremstyle{remark}
\newtheorem{rem}{Remark}
\begin{document}

\bibliographystyle{IEEEtran}
\vspace{3cm}

\title{DISCRETE 11.9.4 Q-1}
\author{EE23BTECH11207 -KAILASH.C$^{*}$% <-this % stops a space
}
\maketitle
\newpage
\bigskip

\renewcommand{\thefigure}{\theenumi}
\renewcommand{\thetable}{\theenumi}


\textbf{QUESTION:}\\
Find the sum of n terms of the series:
$1\times2+2\times3+3\times4+4\times5+....$\\ \\
\textbf{TABLE}\\
\begin{table}[h]
\begin{tabular}{|l|l|}
\hline
\textbf{Symbols} & \textbf{Definition}\\ \hline
$x\brak{n}$ & General term \\ \hline
$y\brak{n}$ & Sum of n terms \\ \hline
$Y\brak{z}$ & Z-Transformation Of $y\brak{n}$\\ \hline
\end{tabular}
\caption{Input Parameters}
\label{Fig:1}
\end{table}\\ \\
\textbf{SOLUTION:}\\
\begin{align}
 x\brak{n}&=\brak{n+1}\brak{n+2}u\brak{n}   \label{eq:1}
\end{align}
By Z-transformation property:
\begin{align}
Z\sbrak{nf\brak{n}}=-z\frac{d}{dz}F\brak{z}\label{eq:2}
\end{align}
By \eqref{eq:2},We have the formulas for:
\begin{align}
n\xleftrightarrow[]{Z-Transformation} \frac{z^{-1}}{\brak{1-z^{-1}}^2}\label{eq:3}\\
n^2\xleftrightarrow[]{Z-Transformation}\frac{z^{-1}\brak{z^{-1}+1}}{\brak{1-z^{-1}}^3},\label{eq:4}\\
n^3\xleftrightarrow[]{Z-Transformation}\frac{z^{-1}\brak{z^{-2}+4z^{-1}+1}}{\brak{1-z^{-1}}^4}\label{eq:5}
\end{align}
Using \eqref{eq:3},\eqref{eq:4} for z-transformation of \eqref{eq:1}
\begin{align}
    X\brak{z}&=\frac{z^{-1}\brak{z^{-1}+1}}{\brak{1-z^{-1}}^3}+\frac{3z^{-1}}{\brak{1-z^{-1}}^2}+\frac{2}{1-z^{-1}},\quad\abs{z}>\abs{1}\label{eq:6}\\
    Y\brak{z}&=X(z)*U(z)\label{eq:7}\\
   &=\brak{\frac{z^{-1}\brak{z^{-1}+1}}{\brak{1-z^{-1}}^3}+\frac{3z^{-1}}{\brak{1-z^{-1}}^2}+\frac{2}{1-z^{-1}}}\frac{1}{1-z^{-1}}\label{eq:8}\\
    &=\frac{z^{-1}\brak{z^{-1}+1}}{\brak{1-z^{-1}}^4}+\frac{3z^{-1}}{\brak{1-z^{-1}}^3}+\frac{2}{\brak{1-z^{-1}}^2}\label{eq:9}
\end{align}
Using contour integration to find inverse Z transformation in \eqref{eq:9}:
\begin{align}
     y\brak{n} & =  \frac{1}{2\pi j} \oint_C Y\brak{z} z^{n-1} dz\label{eq:10}\\
   &=\frac{1}{2\pi j}\oint_{C}\frac{2z^{n+2}}{\brak{z-1}^4}\label{eq:11}
\end{align}
We can observe that pole is repeated 4 times,thus m=4.
\begin{align}
 R&=\frac{1}{\brak {m-1}!}\lim\limits_{z\to z_{0}}\frac{d^{m-1}}{dz^{m-1}}\brak {f\brak {z}}\label{eq:12}\\
 &=\frac{1}{\brak {3}!}\lim\limits_{z\to 1}\frac{d^{3}}{dz^{3}}\brak {z^{n+2}}\label{eq:13}\\
 y\brak{n}&=\frac{n^3+3n^2+2n}{3}\label{eq:14}
\end{align}
\textbf{Graph:}
\begin{figure}[h]
        \centering
\includegraphics[width=\columnwidth]{Graph_of_y(n).png}
\caption{Graph of $y\brak{n}$ vs n}
\label{fig:enter-label}
\end{figure}
\end{document}

