% \iffalse
\let\negmedspace\undefined
\let\negthickspace\undefined
\documentclass[journal,12pt,twocolumn]{IEEEtran}
\usepackage{cite}
\usepackage{amsmath,amssymb,amsfonts,amsthm}
\usepackage{algorithmic}
\usepackage{graphicx}
\usepackage{textcomp}
\usepackage{xcolor}
\usepackage{txfonts}
\usepackage{listings}
\usepackage{enumitem}
\usepackage{mathtools}
\usepackage{gensymb}
\usepackage{comment}
\usepackage{tikz}
\usepackage[breaklinks=true]{hyperref}
\usepackage{tkz-euclide} 
\usepackage{listings}
\usepackage{gvv}
\def\inputGnumericTable{}
\usepackage[latin1]{inputenc}                              
\usepackage{color}                                            
\usepackage{array}                                            
\usepackage{longtable}                                       
\usepackage{calc}                                             
\usepackage{multirow}                                         
\usepackage{hhline}                                           
\usepackage{ifthen}                                           
\usepackage{lscape}

\newtheorem{theorem}{Theorem}[section]
\newtheorem{problem}{Problem}
\newtheorem{proposition}{Proposition}[section]
\newtheorem{lemma}{Lemma}[section]
\newtheorem{corollary}[theorem]{Corollary}
\newtheorem{example}{Example}[section]
\newtheorem{definition}[problem]{Definition}
\newcommand{\BEQA}{\begin{eqnarray}}
\newcommand{\EEQA}{\end{eqnarray}}
\newcommand{\define}{\stackrel{\triangle}{=}}
\theoremstyle{remark}
\newtheorem{rem}{Remark}
\begin{document}

\bibliographystyle{IEEEtran}
\vspace{3cm}

\title{DISCRETE 11.9.4 Q-1}
\author{EE23BTECH11207 -KAILASH.C$^{*}$% <-this % stops a space
}
\maketitle
\newpage
\bigskip

\renewcommand{\thefigure}{\theenumi}
\renewcommand{\thetable}{\theenumi}


\textbf{QUESTION:}\\
Find the sum of n terms of the series:
$1\times2+2\times3+3\times4+4\times5+....$\\ \\
\textbf{SOLUTION:}\\
In the above series,we have:$a_n=(n+1)(n+2)$
\begin{align}
     S&=\sum_{n=0}^{n-1} (n+1) \cdot (n+2)\\
     &=\sum_{n=0}^{n-1} (n^2 + 3n + 2)\\
     &=\sum_{n=0}^{n-1} n^2+3\sum_{n=0}^{n-1} n+2 \sum_{n=0}^{n-1} 1
\end{align}
By the formula for sum of series,we have:
\begin{align}
    \sum_{n=0}^{n-1} n^2&=\frac{(n-1)(n)(2n+1)}{6}\ \\
    \sum_{n=0}^{n-1} n&=\frac{(n-1)(n)}{2}\\
    \sum_{n=0}^{n-1} 1&=n
\end{align}
Using eq-(4),eq-(5) and eq-(6) in eq-(3):
\begin{align}
    S(n)&=\frac{(n-1)n(2n-1)}{6} + \frac{3(n-1)n}{2} + 2n \\
    &=\frac{n(n-1)}{6}[2n-1+9]+2n\\
    &=\frac{n}{6}[(n-1)(2n+8)+12]\\
    &=\frac{n}{6}[2n^2+6n+4]\\
    &=\frac{2n^3+6n^2+4n}{6}\\
    &=\frac{n^3+3n^2+2n}{3}
\end{align}\\
\textbf{Z-Transformation of S(n):}\\
\begin{align}
S(z)&=\frac{1}{3} \sum_{n=0}^{\infty} (n^3 + 3n^2 + 2n) \cdot z^{-n}
\end{align}
By applying z-transformation to each term:
\begin{align}
    S(z)&=\frac{1}{3}\left(\sum_{n=0}^{\infty}n^3 z^{-n}+\sum_{n=0}^{\infty}3n^2z^{-n}+\sum_{n=0}^{\infty}2nz^{-n}\right)
\end{align}
By the z-transformation formulas,we have:
\begin{align}
     Z[n^k]&=-Z\frac{d}{dZ}Z[n^{k-1}]
\end{align}
Using eq-(15) in eq-(14):
\begin{align}
    S(z)&=\frac{1}{3} \left( 1 \sum_{n=0}^{\infty} n^3 \cdot z^{-n} + 3 \sum_{n=0}^{\infty} n^2 \cdot z^{-n} + 2 \sum_{n=0}^{\infty} n \cdot z^{-n} \right)\\
    &=\frac{1}{3} \left( 1 \cdot \frac{z(z^2+4z+1)}{(z-1)^4} + 3 \cdot \frac{z(z+1)}{(z-1)^3} + 2 \cdot \frac{z}{(z-1)^2} \right)\\
    &=\frac{1}{3} \left( \frac{z(z^2+4z+1)}{(z-1)^4} + \frac{3z(z+1)}{(z-1)^3} + \frac{z}{(z-1)^2} \right)
\end{align}
The Z-transformation of eq-(12) is eq-(18) with r.o.c as:$|z|>1$
\newpage
\textbf{Graph of S(n) vs n:}
\begin{figure}[h]
        \centering
\includegraphics[width=1.0\linewidth]{S(n)_plot.png}
    \end{figure}
\end{document}
